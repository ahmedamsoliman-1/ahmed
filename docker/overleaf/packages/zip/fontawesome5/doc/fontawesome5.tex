\documentclass{scrartcl}
\usepackage{hyperref}
\usepackage{shortvrb}
\usepackage{metalogo}
\usepackage{longtable}
\usepackage{fontawesome5}
% \usepackage{xcolor}
% \usepackage[pro]{fontawesome5}
% \faStyle{duotone}
\usepackage[utf8]{inputenc}
\usepackage{geometry}
\MakeShortVerb{\|}
\begin{document}
\title{The fontawesome5 package\thanks{This document corresponds to fontawesome5 version 5.15.4, dated 2022/05/02}}
\author{Font Awesome\thanks{More information at \url{https://fontawesome.com}} (The font)\and Marcel Krüger\thanks{E-Mail: \href{mailto:tex@2krueger.de}{\nolinkurl{tex@2krueger.de}}} (The \LaTeX{} package)}
\maketitle
This package provides \LaTeX{} support for the Font Awesome 5 icons.

To use Font Awesome 5 icons in your document, include the package with
\begin{verbatim}
  \usepackage{fontawesome5}
\end{verbatim}
Alternatively you can add the |fixed| option to get fixed-width icons:
\begin{verbatim}
  \usepackage[fixed]{fontawesome5}
\end{verbatim}
For every icon a macro is provided: Just use the official icon names converted to CamelCase with the prefix |\fa|.
For example to use the |hand-point-up| icon, use |\faHandPointUp|.
For icons ending with |-alt|, append a |*| instead.
An optional argument can be added to select the style (|solid| or |regular|).
The default style is |solid|, it can be changed with |\faStyle{...}|

Every icon can also be accessed using the official icon name. To do this, you can use |\faIcon{the-icon-name}| or |\faIcon[style]{the-icon-name}|.

A list of all included icons with their respective commands can be found at the end of this document.

\subsection*{Example}
\begin{verbatim}
...
\usepackage{fontawesome5}
...
\begin{document}
...
A simple icon: \faHandPointUp\\
Multiple versions of the file icon:
  \faFile~
  \faFile*~
  \faFile[regular]~
  \faFile*[regular].\\
Alternative syntax:
  \faIcon{file}~
  \faIcon*{file}~
  \faIcon[regular]{file}~
  \faIcon*[regular]{file}.
...
\end{document}
\end{verbatim}

A simple icon: \faHandPointUp\\
Multiple versions of the file icon: \faFile~\faFile*~\faFile[regular]~\faFile*[regular].\\
Alternative syntax: \faIcon{file}~\faIcon*{file}~\faIcon[regular]{file}~\faIcon*[regular]{file}.

\subsection*{Font Awesome Pro}
Font Awesome 5 is available in a Free and a Pro version.
This package uses the free version by default.
If you own a Pro license and have the Font Awesome 5 Pro desktop fonts installed in your system font path, you can use Pro instead.
For this, load the package with the |[pro]| option:
\begin{verbatim}
  \usepackage[pro]{fontawesome5}
\end{verbatim}
Now additional icons, the |duotone| and the |light| style can be used.
The second color for |duotone| icons can be set though |\faDuotoneSetSecondary|:
\begin{verbatim}
  \faDuotoneSetSecondary{green}% From here on, the secondary layer
                               % will be green.
\end{verbatim}
Pro is only supported with \XeLaTeX{} and \LuaLaTeX.

\subsection*{Updates}
This package corresponds to Font Awesome 5.15.4.
In case there is a newer version available on the Font Awesome homepage, check for updates on \url{https://ctan.org/pkg/fontawesome5}. Should there be no corresponding update on CTAN, you can write a mail to \href{mailto:tex@2krueger.de}{\nolinkurl{tex@2krueger.de}}.
If you use \XeLaTeX{} or \LuaLaTeX{}, you can also directly download the new Desktop Fonts from \url{https://fontawesome.com} into your \TeX{} tree. If you save them with the filenames\\
{\ttfamily\begin{tabular}{l}
  FontAwesome5Brands-Regular-400.otf\\
  FontAwesome5Free-Regular-400.otf\\
  FontAwesome5Free-Solid-900.otf
\end{tabular}}\\
the package will start using the new version right away.

\subsection*{Bugs}
For bug reports and feature requests, write to \href{mailto:tex@2krueger.de}{\nolinkurl{tex@2krueger.de}}.

\ExplSyntaxOn
\msg_new:nnnn {fontawesome5} {list/no-shorthand} {No~shorthand~defined~for~icon~#1.} {
  It~looks~like~#1~need~special~handling~in~fulllist.tex~but~there~are~
  no~appropriate~definitions.~Ask~a~wizard~to~add~#1~to~fulllist.tex~to~
  fix~this.
}
\tl_new:N \g__fontawesome_last_name_tl
\tl_new:N \g__fontawesome_last_cs_tl
\prg_new_protected_conditional:Nnn \__fontawesome_if_regular_style:nn {T} {
  \group_begin:
    \usefont{U}{fontawesome#1}{regular}{n}
    \iffontchar\font#2
      \group_insert_after:N \prg_return_true:
    \else:
      \group_insert_after:N \prg_return_false:
    \fi:
  \group_end:
}
\tracingonline1
\showboxdepth\maxdimen
\showboxbreadth\maxdimen
\cs_new:Nn\__fontawesome_list_show_icon:nnnn{
  \str_if_in:nnT{#3}{brands}{
    \hfilneg\vbox to0.875em{\vfil\hbox to0pt{\hss\tiny\faTrademark\quad}\vfil}\hfil
  }
  \faIcon{#2}&\texttt{\textbackslash#1}&\texttt{\textbackslash faIcon\{#2\}}
  \str_if_in:nnT{#3}{free}{
    \__fontawesome_if_regular_style:nnT {#3} {#4} {
      \\\faIcon[regular]{#2}&\texttt{\textbackslash#1[regular]}&\texttt{\textbackslash faIcon[regular]\{#2\}}
    }
  }
  \tl_gset:Nn \g__fontawesome_last_cs_tl {#1}
  \tl_gset:Nn \g__fontawesome_last_name_tl {#2}
  \\
}
\cs_generate_variant:Nn \__fontawesome_list_show_icon:nnnn { fnnn }
\cs_set:Nn\__fontawesome_def_icon:nnnnn{
  \__fontawesome_list_show_icon:fnnn{
    \tl_if_empty:nTF{#1}{
      \str_if_eq:noTF{#2}{\g__fontawesome_last_name_tl-alt}{
        \g__fontawesome_last_cs_tl*
      }{
        \cs_if_exist_use:cF {__fontawesome_list_real_cs_#2_tl} {
          \msg_expandable_error:nnn {fontawesome5} {list/no-shorthand} {#2}
        }
      }
    }{
      \cs_to_str:N #1 % You might have noticed that #1 is a n-type argument, not N-type.
      % This is not a mistake, the argument might contain additional characters after the initial cs
      % which is passed to \cs_to_str:N
    }
  }{#2}{#3}{#4}
}
% Some icons are special for some reason. See fontawesome5.sty if you want to know why
% they are singled out.
\tl_const:cn{__fontawesome_list_real_cs_wifi_tl}{faWifi}
\tl_const:cn{__fontawesome_list_real_cs_dice-d20_tl}{faDiceD20}
\tl_const:cn{__fontawesome_list_real_cs_dice-d6_tl}{faDiceD6}
\tl_const:cn{__fontawesome_list_real_cs_signal_tl}{faSignal}
\tl_const:cn{__fontawesome_list_real_cs_stopwatch_tl}{faStopwatch}
\tl_const:cn{__fontawesome_list_real_cs_stopwatch-20_tl}{faStopwatch20}
%
\tl_const:cn{__fontawesome_list_real_cs_500px_tl}{faIcon\{500px\}}
\tl_const:cn{__fontawesome_list_real_cs_arrows-alt_tl}{faArrows*}
\tl_const:cn{__fontawesome_list_real_cs_cloud-download-alt_tl}{faCloudDownload*}
\tl_const:cn{__fontawesome_list_real_cs_cloud-upload-alt_tl}{faCloudUpload*}
\tl_const:cn{__fontawesome_list_real_cs_exchange-alt_tl}{faExchange*}
\tl_const:cn{__fontawesome_list_real_cs_expand-arrows-alt_tl}{faExpandArrows*}
\tl_const:cn{__fontawesome_list_real_cs_external-link-alt_tl}{faExternalLink*}
\tl_const:cn{__fontawesome_list_real_cs_external-link-square-alt_tl}{faExternalLinkSquare*}
\tl_const:cn{__fontawesome_list_real_cs_level-down-alt_tl}{faLevelDown*}
\tl_const:cn{__fontawesome_list_real_cs_level-up-alt_tl}{faLevelUp*}
\tl_const:cn{__fontawesome_list_real_cs_pencil-alt_tl}{faPencil*}
\tl_const:cn{__fontawesome_list_real_cs_shield-alt_tl}{faShield*}
\tl_const:cn{__fontawesome_list_real_cs_sign-in-alt_tl}{faSignIn*}
\tl_const:cn{__fontawesome_list_real_cs_sign-out-alt_tl}{faSignOut*}
\tl_const:cn{__fontawesome_list_real_cs_square-root-alt_tl}{faSquareRoot*}
\tl_const:cn{__fontawesome_list_real_cs_tachometer-alt_tl}{faTachometer*}
\tl_const:cn{__fontawesome_list_real_cs_ticket-alt_tl}{faTicket*}
%
\tl_const:cn{__fontawesome_list_real_cs_compress-arrows-alt_tl}{faCompressArrows*}
\ExplSyntaxOff
\newgeometry{textwidth=18cm}
\subsection*{Full~icon~list~for~FontAwesome~5~Free}
All icons marked with \vbox to0.875em{\vfil\hbox{\hss\tiny\faTrademark}\vfil} are brand icons.
\begin{quote}
  All brand icons are trademarks of their respective owners. The use of these
  trademarks does not indicate endorsement of the trademark holder by Font
  Awesome, nor vice versa. \emph{Please do not use brand logos for any purpose except
  to represent the company, product, or service to which they refer.}
\end{quote}
\ExplSyntaxOn
\begin{longtable}{cll}
  \cs:w @@input\cs_end: fontawesome5-mapping.def~
\end{longtable}
\ExplSyntaxOff
\restoregeometry

\end{document}
